\section{Related Work}
\label{sec:rel}
There is a huge body of work on identifying and fighting black hat SEO, specifically,
spam \cite{AIRWeb,Castillo+Davison:10a}. Our 
approach is essentially a content-based white hat SEO method intended to promote
{\em legitimate} documents in rankings via {\em legitimate}
content modifications. We are not aware of past work on devising such
automatic content modification procedures. 

Our approach might seem at first glance conceptually similar to
the black hat SEO {\em stitching} technique \cite{Gyongyi+Molina:05a}:
authors of low-quality Web pages manually ``glue'' to their
documents unrelated phrases from other documents. In contrast, our approach operates on descent quality
documents and is optimized to maintain document quality.

Our approach can conceptually be viewed as ranking-incentivized
paraphrasing: modifying the document to promote
it in rankings, but keeping content quality and having the content remain faithful to the original content. Past work on paraphrasing
\cite{Androutsopoulos+Malakasiotis:10a} does not include methods
intended to promote documents in ranking.
%Yet, our approach of
%replacing a passage with another is conceptually inspired by the general line of
%work on passage-based paraphrasing (e.g., \cite{Barzilay+Lee:03a}).

We use simple estimates (e.g., lexical and Word2Vec similarities) to
measure the extent to which content coherence is maintained given the passage
replacement. More evolved estimates can be used to this end \cite{graesser2004coh,lapata2005automatic,pitler2008revisiting,lin2011automatically,li2016neural}. Furthermore, one could modify the document using text-generation approaches that account for coherence \cite{kiddon2016globally,ji2016latent,guo2018long}, which we leave for future work.

\endinput

There is a line of work on query-biased/focused content summarization
(e.g., \cite{Tombros+Sanderson:98a,Berger+Mittal:00a}). The main goal
of these methods is to create snippets that can help users of search
engines understand the relevance of retrieved documents to their
queries. In contrast, our approach does not create a short summary of
the given document, but rather modifies it for rank-promotion while maintaining coherence.

%Recent work analyzes the content-based modification strategies of
%documents' authors for promoting their documents in rankings
%\cite{Raifer+al:17a}. The merits of mimicking content from documents
%highly ranked in the past for queries of interest were theoretically
%and empirically demonstrated. Our approach of replacing a passage in a document with a passage from a document highly ranked in the past for a query of interest is inspired by this strategy.




